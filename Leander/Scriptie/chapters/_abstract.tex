This thesis proposes a novel approach to Music Structure Analysis (MSA). This approach implements the Segmentation by Annotation (SbA) approach to MSA, using a convolutional neural network (CNN) and an artificial neural network using Long Short-Term Memory (LSTM) units. An overview of the current advances in music structure analysis is given as well as the use of the proposed architectures in similar research fields. A description of the testing methods is provided in which the proposed architectures show promising results on the custom ground truth used. This custom ground truth is a modified version of the segment function annotations found in the internet archives subset of the SALAMI dataset. The ground truth is modified by reducing the amount of unique functions from 26 to 9 because of the low occurrence of some labels in the original data and to improve the accuracy of the machine learning models. Due to this custom ground truth, comparison with the current state-of-the-art of music structure analysis proved quite hard. By comparing the SbA approach to the (more symbolic) Distance-based Segmentation and Annotation approach, a comparison between using machine learning and non-machine learning techniques can be made. Future research is proposed to enhance the segmentation by annotation approach as well as music structure analysis in general.\\
\\
\textbf{Keywords:} Music Structure Analysis, Music Information Retrieval, Segmentation by Annotation, Convolutional Neural Networks, Long Short-Term Memory